\documentclass[11pt,a4paper]{article}

% document properties

\begin{document}
\section{Problems}
\subsection{Supervised Learning}
\paragraph*{Def:} Give the algorithm a dataset in which the right answers are given. Algorithm then can produce more right answers. 

Learning can be used to solve two types of problems:
\begin{itemize}
\item \textbf{Regression:} predict a continuous value output.
\item \textbf{Classification:} predict a discrete output, ie. what label to attribute to the data.
\end{itemize}

When learning, a data point can have multiple \emph{features}. A data point is basically a vector of features, with each feature either a scalar or a vector itself.

\subsection{Unsupervised Learning}
\paragraph*{Def:} The algorithm has to discover structures in the data on its own, without input from the user. This is usually done by \emph{clustering}. Clustering is used for example to detect patterns in news headlines, friend groups in a social network, or even on star positions in space.

\end{document}